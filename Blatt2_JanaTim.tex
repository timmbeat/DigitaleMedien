\documentclass[11pt,twoside,a4paper]{article}
\usepackage[utf8]{inputenc}
\usepackage{algorithm}
\usepackage{listings}


\begin{document}

\section{Digitale Medien Blatt 2 }

\subsection{Aufgabe 1}

p(t) = p_3^3  = (1-t) p^2_2 + t p^2_3 

= (1-t)[(1-t) p_1^1 + t p_2^1] + t[(1-t)p^1_2 + t p^1_3]

= (1-t)[(1-t)[(1-t)p_0 + t p_1]+t[(1-t) p_1 + t p_2]] + t[(1-t)[(1-t) p_1 + t p_2] + t [(1-t) p_2 + t p_3]]

= (1-t)[(1-t)^2 p_0 + t(1-t)p_1 + t(1-t)p_1 + t^2 p_2] + [(1-t)^2 p_1 + t(1-p) p_2 + t(1-t)p_2 + t^2 p_3]

= (1-t)^{3} p_0 + t(1-t)^{2} p_1 + t(1-t)^2 p_1 + t^2 (1-t)p_2 + t(1-t)^2 p_1 + t^2 (1-t) p_2 + t^2 (1-t) p_2 + t^3 p_3

\subsection{Aufgabe 2}

n = Ordnung 

\displaystyle\sum_{i=0}^{n} ((n-k+1) t^k (1-t)^{(n-k)} p	^k ) - (n+1)


\subsection{Aufgabe 3}

\begin{algorithm}[H]
\caption{Einfacher Algorithmus}\label{euclid}
\begin{lstlisting} [mathescape=true]
$t_{start}$ = 0
n = Dimension
while($t_{start}$ < 1){
    for(k = 0, k<n, i++){
    	P = P + (n-k+1) $t_{start}^k$ $(1-t_{start})^{(n-k)}$ $p^k$
    }
    P - (n+1)
    draw P($t_{start}$)
    $t_{start}$ += $\varepsilon $
}
\end{lstlisting}
\end{algorithm}


\begin{algorithm}[H]
\caption{Bezier Algorithmus}\label{euclid}
\begin{lstlisting}[mathescape=true]
function devide(point P0, point P1, point P2, point P3){
	if(isLine(P0,P1,P2,P3) == true){
		drawLine(point p0, point p1)
	}else{
		devide(point P0, point P1)
		devide(point P2, point P3) 
	}
}
\end{lstlisting}
\end{algorithm}




\subsection{Aufgabe 4}

Algorithmus 1:

\begin{itemize} 
\item (+) einfach zu implementieren 
\item (-) viele Berechnungen: lange Laufzeit 
\item (-) trotzdem können große Abstände zwischen manchen Punkten liegen, sodass die Linie nicht durchgängig ist.
\end{itemize}


Algorithmus 2: 

\begin{itemize} 
\item (+) keine Pixel-Lücken in der Kurve
\item (+) bessere Laufzeit wegen divide and conquer
\end{itemize}




\end{document}